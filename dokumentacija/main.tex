\documentclass[12pt, a4paper]{article}
\usepackage{graphicx} % Required for inserting images
\usepackage[final]{pdfpages}
\usepackage[serbian]{babel} % Use the Serbian language package
\usepackage{fontspec} % Required for using system fonts
\usepackage{geometry} % Required for setting page size
\usepackage{fancyhdr} % Required for custom headers and footers
\usepackage[scaled]{helvet}
\usepackage{csquotes}
\usepackage{chngcntr}
\usepackage[colorlinks=true,linkcolor=black,citecolor=black,urlcolor=black]{hyperref} % Customize link colors
\usepackage[fixlanguage]{babelbib} \bibliographystyle{babunsrt}
\usepackage{setspace}
\usepackage{tocloft}
\usepackage{minted} % Paket za umetanje koda
\usepackage{listings}
\usepackage{subcaption}

% \setmainfont{Nimbus Sans L}
\setmainfont{Roboto}
\setlength{\parindent}{0pt}
\setlength{\parskip}{12pt}%

\geometry{a4paper, margin=1in}

% Customize headers and footers
\pagestyle{fancy}
\fancyhf{} % Clear header and footer
\fancyfoot[R]{\thepage} % Right align page number in the footer
\renewcommand{\headrule}{} % Remove header line

\counterwithin{figure}{section}
\addto\captionsserbian{\renewcommand{\figurename}{Слика}}

\setlength{\cftbeforesecskip}{4pt} % Razmak pre svake sekcije
\setlength{\cftparskip}{2pt} % Razmak između stavki u sadržaju

\definecolor{codegreen}{rgb}{0,0.6,0}
\definecolor{codegray}{rgb}{0.5,0.5,0.5}
\definecolor{codepurple}{rgb}{0.58,0,0.82}
\definecolor{backcolour}{rgb}{0.95,0.95,0.92}


\begin{document}

\lstdefinestyle{mystyle}{
    backgroundcolor=\color{backcolour},   
    commentstyle=\color{codegreen},
    keywordstyle=\color{magenta},
    numberstyle=\tiny\color{codegray},
    stringstyle=\color{codepurple},
    basicstyle=\ttfamily\footnotesize,
    breakatwhitespace=false,         
    breaklines=true,                 
    captionpos=b,                    
    keepspaces=true,                 
    numbers=left,                    
    numbersep=5pt,                  
    showspaces=false,                
    showstringspaces=false,
    showtabs=false,                  
    tabsize=2
}


\includepdf[pages=-]{prva_strana.pdf}

\renewcommand{\contentsname}{Садржај}
\tableofcontents
\pagebreak

\section{Увод}
\textit{Blockchain} технологија представља дистрибутивну, децентрализовану и јавну базу свих трансакција \cite{1}.


Први концепт \textit{blockchain} технологије помиње се у 1982. години, када је Давид Чаум у својој дисертацији описао дистрибуирану базу података која користи криптографију \cite{2}. Овај рани рад није био директно повезан са дигиталним валутама, али је поставио темеље за будући развој \textit{blockchain} - а.

Права револуција долази 2008. године када Сатоши Накамото објављује рад "\textit{Bitcoin}: \textit{Peer-to-peer} електронски готовински систем", уводећи први модерни \textit{blockchain} и криптовалуту \textit{Bitcoin}. Генесис блок, први блок \textit{Bitcoin blockchain} - а, ископан је 3. јануара 2009. године, означавајући почетак \textit{blockchain} технологије какву данас познајемо \cite{3}.

\textit{Etherium}, лансиран 2015. године од стране Виталика Бутерина, увео је паметне уговоре који омогућавају сложеније трансакције и аутоматизацију различитих процеса. Овај развој проширио је примену \textit{blockchain} технологије далеко изван дигиталних валута, омогућавајући креирање децентрализованих апликација \cite{4}.

\textit{Blockchain} технологије су се од свог настанка имплементирале у различитим програмским језицима и окружењима. У својим раним фазама, \textit{blockchain} технологије су се углавном развијале користећи језике као што су \textit{C++} и \textit{Java}, захваљујући њиховој ефикасности и широкој употреби у индустрији. Касније, с појавом паметних уговора, \textit{Solidity} је постао стандард за развој на \textit{Etherium} платформи.

Овај рад се фокусира на имплементацију основних концепата \textit{blockchain} технологије у програмском језику \textit{Rust}, који је познат по својој сигурности, перформансама и могућности превенције грешака при руковању меморијом.

Рад је структуиран X целина
\pagebreak

\section{Основе \textit{Rust} програмског језика}
\textit{Rust} је савремени програмски језик који је развијен да буде безбедан и брз. Развијен од стране \textit{Mozilla Research}-а, \textit{Rust} је од свог настанка привукао велику пажњу због својих изузетних безбедносних карактеристика и перформанси \cite{5}.

\subsection{Увод у \textit{Rust} програмски језик}
\textit{Rust} је системски програмски језик, а уместо интерпретираног језика, као што су \textit{JavaScript} или \textit{Ruby}, има компајлер, као што имају \textit{Go}, \textit{C} или \textit{Swift}. Не комбинује активни \textit{runtime}, али обезбеђује језичку ергономију. Све је ово могуће захваљујући компајлеру који спречава грешке било којег типа и осигурава да не дође до проблема у меморији пре него што се покрене апликација \cite{6}.

\textit{Rust} обезбеђује перформансе (нема \textit{runtime}, нити прикупљање "смећа"), безбедност (компајлер осигурава да је све безбедно за меморију, чак и у асинхроним окружењима) и продуктивност (његове уграђене алатке за тестирање, документацију и "менаџер" пакета чине га лаким за израдз и одржавање) \cite{6}. 

\subsection{Зашто \textit{Rust} за \textit{blockchain}?}
Када је у питању развој blockchain апликација, \textit{Rust} се истиче као одличан избор из неколико разлога:
\begin{enumerate}
    \item \textbf{Безбедност меморије}: \textit{Rust}-ов систем власништва и провера за време компилације осигуравају да програмери избегну уобичајене грешке у раду са меморијом, што је критично за сигурност \textit{blockchain} система \cite{7}.
    \item \textbf{Перформансе}: \textit{Rust} је дизајниран да буде брз и ефикасан. Његов минималан \textit{overhead} и високо оптимизован компајлер резултирају брзим извршавањем кода, што је важно за обраду великог броја трансакција у реалном времену \cite{7}.
    \item \textbf{Паралелизам и конкурентност}: \textit{Rust} нуди снажну подршку за паралелно и конкурентно програмирање, омогућавајући оптимално коришћење мулти-језгарних процесора \cite{7}.
\end{enumerate}

\textit{Rust} нуди низ алата и библиотека које олакшавају развој сложених апликација. Две од најзначајнијих библиотека за развој \textit{blockchain} апликација су \textit{Tokio} и \textit{libp2p}. Ове библиотеке пружају подршку за асинхроне позиве и \textit{peer-to-peer} комуникацију, што је кључно за функционалност и ефикасност \textit{blockchain} система.

\pagebreak

% \textbf{\textit{Tokio}} је моћна асинхрона \textit{runtime} библиотека за \textit{Rust} која омогућава развој високоперформансних и високо доступних апликација. Кроз \textit{Tokio}, програмери могу да имплементирају асинхроне позиве и да развију веб сервере који могу да обрађују велики број истовремених веза.

Слика \ref{fig:2.1} прииказује технички стек који је укључен у одабир радног оквира. \textbf{\textit{Warp}} је довољно мали да се "склони са пута", довољно се користи да се њиме управља активно и има активну заједницу. Заснован је на \textit{Tokio runtime} - у. 


\begin{figure}[h]
    \centering
    \includegraphics[width=1\linewidth]{slike/warp.png}
    \caption{\textit{Warp} веб радни оквир}
    \label{fig:2.1}
\end{figure}

\textit{\textbf{libp2p}} је модуларни мрежни стек који омогућава \textit{peer-to-peer} комуникацију. У контексту \textit{blockchain}-а, \textit{libp2p} се користи за омогућавање комуникације између различитих чворова у мрежи. Ова библиотека је флексибилна и подржава различите протоколе за пренос података, што је чини идеалном за развој децентрализованих апликација.

\pagebreak

\section{Увод у \textit{blockchain} технологију}
\textit{Blockchain} технологија представља савремен приступ складиштењу и дистрибуцији података. Основни принципи и концепти \textit{blockchain} технологије нуде дубоку промену у начину на који се информације похрањују, проверавају и дистрибуирају путем децентрализоване мреже рачунара.

\subsection{Основни принципи и концепти}
\textit{Blockchain} се може дефинисати као дистрибуисана дигитална књига трансакција. Основна идеја је стварање низа блокова који садрже податке. Блкови су криптографски повезани тако да је немогуће мењати податке у претходним блоковима без мењања свих следећих блокова \cite{8}. 

Кључни елементи \textit{blockchain}-а укључују:
\begin{enumerate}
    \item \textbf{Децентрализација}: Подаци се похрањују и управљају путем мреже чворова уместо централизованог ауторитета, што осигурава транспарентност и отпорност на цензуру.
    \item  \textbf{Дистрибуираност}: Сваки чвор у мрежи садржи комплетан или део \textit{blockchain}-а, омогућујући свима у мрежи да виде исте податке. Ово спречава појединачне тачке квара и повећава отпорност на нападе.
    \item \textbf{Сигурност}: Криптографски алгоритми осигуравају да је свака промена у \textit{blockchain}-у лако проверљива, а трансакције се потврђују кроз консензус мреже.
    \item \textbf{Неповратност}: Након што је трансакција забележена у \textit{blockchain}-у, тешко ју је променити или обрисати без сагласности већине чворова у мрежи, чиме се осигурава поверење и интегритет података.
\end{enumerate}


\subsection{Поређење са традиционалним базама података}
Насупрот традиционалним базама података које су често централизоване и ослањају се на поверење у једну јединицу, \textit{blockchain} нуди неколико кључних разлика:
\begin{enumerate}
    \item \textbf{Централизација у односу на децентрализацију}: Традиционалне базе података често су централизоване под контролом једне организације. \textit{Blockchain} дистрибуише податке широм мреже, елиминишући потребу за централним ауторитетом.
    \item \textbf{Транспарентност и проверљивост}: \textit{Blockchain} омогућава свим корисницима увид у све трансакције које су се догодиле, што повећава транспарентност и смањује могућност манипулације.
    \item \textbf{Сигурност и отпорност}: Због своје дистрибуиране природе, \textit{blockchain} је отпорнији на нападе и кварове у поређењу са традиционалним базама података које су осетљиве на појединачне тачке квара.
    \item \textbf{Ефикасност и трошкови}: Иако \textit{blockchain} може бити спорији у обради података у поређењу са централизованим базама података, његова сигурност и транспарентност могу надмашити трошкове и ризике традиционалних система.
\end{enumerate}

\newpage
\section{Архитектура апликације}
\textit{Blockchain} технологија се састоји од неколико кључних компоненти које омогућавају њено функционисање. Основне јединице података су блокови, који садрже информације о трансакцијама, временским ознакама и криптографским хеш функцијама претходних блокова. Ови блокови су повезани у секвенцијални ланац, познат као \textit{blockchain}, који осигурава неповредивост података. 

Дистрибуирана мрежа чворова заједнички одржава и верификује \textit{blockchain}, омогућавајући децентрализацију. Консензус алгоритми, као што су \textit{Proof of Work (PoW)} и \textit{Proof of Stake (PoS)}, омогућавају учесницима мреже да се сложе око валидности нових блокова. Криптографија осигурава сигурност и приватност података унутар \textit{blockchain} -а, користећи хеш функције и дигиталне потписе. 

У наредним подсецијама, детаљно ћемо описати сваку од ових компоненти, укључујући процесе као што су PoW и мајнинг, који су кључни за додавање нових блокова у ланац.


\subsection{Блокови}
Блокови су основне јединице података у \textit{blockchain} технологији. Сваки блок садржи скуп података који су повезани са трансакцијама и другим важним информацијама. У контексту \textit{blockchain}-а, блокови су организовани у ланац, где сваки блок садржи хеш претходног блока, што обезбеђује интегритет и сигурност података. Следећи код приказује структуру блока у Rust програмском језику:

\begin{minted}{rust}
pub struct Block {
    pub timestamp: DateTime<Utc>,
    pub last_hash: String,
    pub hash: String,
    pub data: Vec<Transaction>,
    pub nonce: u64,
    pub difficulty: u64,
}
\end{minted}

Атрибути блока су:
\begin{itemize}
    \item \textbf{timestamp}: Време када је блок креиран. Овај атрибут омогућава праћење хронологије трансакција у \textit{blockchain}-у.
    \item \textbf{last\_hash}: Хеш вредност претходног блока у ланцу. Овај атрибут обезбеђује да сваки блок буде повезан са својим претходником, чиме се осигурава интегритет ланца.
    \item \textbf{hash}: Хеш вредност тренутног блока. Ова вредност се добија применом хеш функције на садржај блока и служи као јединствени идентификатор блока.
    \item \textbf{data}: Податке у блоку, који обично укључују трансакције. У овом случају, то је вектор трансакција \textit{(Vec<Transaction>)}.
    \item \textbf{nonce}: Произвољни број који рудари мењају током процеса рударења како би добили хеш вредност блока која задовољава критеријуме тешкоће.
    \item \textbf{difficulty}: Ниво тежине који одређује колико је сложено пронаћи важећи хеш за блок. Тежина рударења се прилагођава да би се одржала константна брзина креирања блокова у мрежи.
\end{itemize}



\subsubsection{Генесис блок}
Генесис блок је први блок у ланцу блокова и служи као темељ целокупног \textit{blockchain} система (Слика \ref{fig:genesis-block}). Он нема претходника и обично је ручно креиран од стране креатора \textit{blockchain}-а. Генесис блок обично садржи посебне параметре и почетне вредности које су специфичне за дат \textit{blockchain}. Његова важност лежи у чињеници да сваки наредни блок у ланцу зависи од њега кроз хеш вредности.

\begin{figure}[h]
    \centering
    \includegraphics[width=1\linewidth]{slike/genesis.png}
    \caption{Приказ генесис блока у \textit{blockchain}-у}
    \label{fig:genesis-block}
\end{figure}

\subsubsection{Рударење}
Рударење је процес додавања нових блокова у \textit{blockchain}. Рудари користе своју рачунарску снагу да реше комплексне математичке проблеме који су потребни за валидацију нових трансакција и креирање нових блокова. Овај процес захтева значајну количину енергије и ресурса, али је кључан за одржавање безбедности и децентрализације \textit{blockchain} мреже. У процесу рударења, рудари се такмиче да пронађу одговарајући \textit{nonce} који ће произвести хеш вредност која испуњава одређене критеријуме тешкоће.


\subsubsection{Хеш функција}
Хеш функција је критичан елемент у \textit{blockchain} технологији, јер обезбеђује сигурност и интегритет података у блоковима. Хеш функција узима улазне податке произвољне дужине и генерише фиксну дужину излазне вредности, која је јединствена за те улазне податке. У контексту \textit{blockchain}-а, хеш функција се користи да повезује сваки блок са претходним блоком, чиме се обезбеђује да свака промена у подацима било ког блока одмах утиче на све наредне блокове, што чини \textit{blockchain} изузетно отпорним на манипулацију.



\subsection{Ланац}
\textit{Blockchain} је структура података која се састоји од низа повезаних блокова, где сваки блок садржи хеш претходног блока, чиме се обезбеђује интегритет и сигурност ланца. Следећи код приказује структуру \textit{blockchain}-а у \textit{Rust} програмском језику:

\begin{minted}{rust}
pub struct Blockchain {
    pub chain: Vec<Block>,
}
\end{minted}

Атрибут \textit{\textbf{chain}} је вектор који чува редоследно повезане блокове, формирајући ланац блокова.
Слика \ref{fig:genesis-blockchain} приказује структуру ланца са генесис блоком на почетку.
\begin{figure}[h]
    \centering
    \includegraphics[width=1\linewidth]{slike/blockchain.png}
    \caption{Приказ идеје \textit{blockchain}-a}
    \label{fig:genesis-blockchain}
\end{figure}

\newpage
\subsubsection{Валидација више ланаца}
Идеја овог механизма је да подржи више доприносиоца, при чему ће више доприносиоца додавати блокове у \textit{blockchain}. Сваки рудар ће имати своју верзију истог ланца. Када један рудар дода нови блок у ланац, мораће да пошаље тај нови блок осталим ланцима у систему како би они прихватили ту промену и ажурирали целокупни систем. На тај начин сви добијају ажурирану копију са тим новим блоком, чиме се осигурава да сви ланци буду конзистентни.


\begin{figure}[h]
    \centering
    \includegraphics[width=1\linewidth]{slike/multiple-chain-validation.png}
    \caption{Приказ дељења ланаца}
    \label{fig:multiple-chain-validation}
\end{figure}

Међутим, да би сви рудари прихватили ове нове ланце, мора постојати неки облик валидације који ће осигурати да је нови блок валидан и да треба да буде прихваћен. Главни облик валидације је прихватање дужих ланаца који стигну. На пример, ако сви имају договорени \textit{blockchain} који је већ дуг три блока, и један рудар дода два блока у ланац, док други рудар дода само један блок у исто време, систем ће прихватити дужи ланац. На тај начин се осигурава да договорени ланац за све увек буде онај који садржи највише података.


\begin{figure}[h]
    \centering
    \begin{minipage}{0.45\linewidth}
        \centering
        \includegraphics[width=\linewidth]{slike/longer-chains-before.png}
        \caption{Стање пре валидације}
        \label{fig:validation-1}
    \end{minipage}
    \hfill
    \begin{minipage}{0.45\linewidth}
        \centering
        \includegraphics[width=\linewidth]{slike/longer-chains-after.png}
        \caption{Стање после валидације}
        \label{fig:validation-2}
    \end{minipage}
\end{figure}

\newpage
Ово такође решава проблем рачвања у ланцу. На пример, ако две одвојене инстанце \textit{blockchain}-а истовремено произведу један блок на основу претходног блока, настаје рачвање у систему где оба рудара производе блок на основу истог претходног блока (слика \ref{fig:forks-a}). Пола рудара ће имати ланац који је произвео рудар А, а друга половина ће имати ланац који је произвео рудар Б (слика \ref{fig:forks-b}). Коначно, систем треба да дође до договора о томе који ланац ће прихватити. Ако неко дода неколико блокова на ланац рудара А, тај ланац ће сада бити дужи од свих осталих у систему (слика \ref{fig:forks-c}). Сви ће морати да прихвате најдужи ланац, који садржи блок од рудара А, чиме се решава рачвање прихватањем оригиналног блока од рудара А (слика \ref{fig:forks-d}).

\begin{figure}[h]
    \centering
    \begin{subfigure}{0.45\linewidth}
        \includegraphics[width=\linewidth]{slike/forks-1.png}
        \caption{Додавање различитих блокова}
        \label{fig:forks-a}
    \end{subfigure}
    \hfill
    \begin{subfigure}{0.45\linewidth}
        \includegraphics[width=\linewidth]{slike/forks-2.png}
        \caption{Размена различитих ланаца}
        \label{fig:forks-b}
    \end{subfigure}

    \begin{subfigure}{0.45\linewidth}
        \includegraphics[width=\linewidth]{slike/forks-3.png}
        \caption{Додавање нових блокова}
        \label{fig:forks-c}
    \end{subfigure}
    \hfill
    \begin{subfigure}{0.45\linewidth}
        \includegraphics[width=\linewidth]{slike/forks-4.png}
        \caption{Прихватање најдужег ланца}
        \label{fig:forks-d}
    \end{subfigure}
    \caption{Решавање рачвања}
    \label{fig:combined}
\end{figure}

Ово не значи да блокови са ланца рудара Б губе оригиналне податке, јер се блок који није укључен у рачвање може сада додати на крај новоприхваћеног ланца.

Други облик валидације је провера вредности хеша произведених за сваки блок ланца. Сваки \textit{blockchain} има приступ хеш функцији која генерише хеш на основу података блока. Када \textit{blockchain} прими нови ланац, може осигурати да је хеш исправно генерисан тако што ће сам поново генерисати тај хеш. Ако се хешеви не поклапају, вероватно су подаци мењани, и због тога \textit{blockchain} неће прихватити нови ланац.


% \subsection{\textit{HTTP} сервер}

% \subsection{\textit{P2P} сервер}

% \subsection{\textit{Proof of work}}
% \subsubsection{\textit{Nonce}}
% \subsubsection{Динамичка тежина додавања блокова}

% \subsection{Новчаник и трансакције}
% \subsubsection{Новчаник}
% \subsubsection{Креирање кључева}
% \subsubsection{Трансакције}
% \subsubsection{Потписивање трансакција}
% \subsubsection{Верификација трансакција}
% \subsubsection{Ажурирање трансакција}

% \subsection{Базен трансакција}

% \subsection{Рудари}
% \subsubsection{Добављање валидне трансакције}
% \subsubsection{Наградне трансакције}
% \subsubsection{Пражњење базена трансакција}
% \subsubsection{Стање новчаника}

% \section{Ограничења и унапређења}
% \section{Закључак}



\pagebreak
\section{Литература}
\renewcommand{\refname}{}
\vspace{-\parskip} % Remove extra space added by \parskips
\vspace{-\parskip} % Remove extra space added by \parskips
\vspace{-\parskip} % Remove extra space added by \parskips
\vspace{-\parskip} % Remove extra space added by \parskips
\bibliography{bibliography}


\pagebreak
\section{Подаци о кандидату}
Кандидат Бојан Мијановић је рођен 2002. године у Зрењанину. Завршио је средњу школу у Зрењанину, 2020. године као ђак генерације. Факултет Техничких Наука у Новом Саду је уписао 2020. године. Испунио је све обавезе и положио је све испите предвиђеним студијским програмом са просечном оценом од 9.75.


\end{document}

